\documentclass[a4paper,12pt]{extarticle}
\usepackage[utf8x]{inputenc} 
\usepackage[left=2.5cm, top=2cm, right=1cm, bottom=20mm, nohead, nofoot]{geometry}
\usepackage[T1,T2A]{fontenc}
\usepackage[russian]{babel}
\usepackage{hyperref}
\usepackage{indentfirst}
\usepackage{listings}
\usepackage{color}
\usepackage{here}
\usepackage{array}
\usepackage{multirow}
\usepackage{graphicx}
\usepackage{amsmath}
\usepackage{amssymb}
\begin{document}
	\begin{center}		% выравнивание по центру

		\large Санкт-Петербургский политехнический университет Петра Великого\\
		\large Институт компьютерных наук и технологий \\
		\large Кафедра компьютерных систем и программных технологий\\[5cm]
		% название института, затем отступ 5см
		
		\huge Сети и телекоммуникации\\[0.5cm] % название работы, затем отступ 0,5см
		\large Отчет по лабораторной работе\\[0.1cm]
		\large "Сетевые технологии"\\[4cm]

	\end{center}
	\begin{flushright} % выравнивание по правому краю
		\begin{minipage}{0.35\textwidth} % врезка в половину ширины текста
			\begin{flushleft} % выровнять её содержимое по левому краю

				\large\textbf{Работу выполнил:}\\
				\large Легун Б.А.\\
				\large {Группа:} 43501/1\\
				
				\large \textbf{Преподаватель:}\\
				\large Алексюк А.О.\\

			\end{flushleft}
		\end{minipage}
	\end{flushright}
	
	\vfill % заполнить всё доступное ниже пространство

	\begin{center}
	\large Санкт-Петербург\\
	\large \the\year % вывести дату
	\end{center} % закончить выравнивание по центру

\thispagestyle{empty} % не нумеровать страницу
\newpage
\section{Цель работы}
Разработать приложение «Сетевой форум».
Платформа сервера: Windows.
Платформа клиента: Unix.

\section{Реализованые функции}
\begin{enumerate}
\item Получения списка команд 		
\item Создание нового форума 			 
\item Отключения от сервера 			
\item Получение списка доступных форумов
\item Открытие форума 				
\item Выход с форума
\item Авторизация 		
\item Получения списка пользователей онлайн
\item Получение иерархии форума		
\end{enumerate}

\section{Методы обмена сообщениями}
\begin{itemize}
\item SendBytes() :
\begin{itemize}
\item SendInt() - отправка длины сообщения
\item send() - отправка самого сообщения
\end{itemize}
\item ReadBytes() :
\begin{itemize}
\item ReadInt() - чтение ожидаемой длинны сообщения
\item recv() - чтение самого сообщения
\end{itemize}
\end{itemize}
\section{Формат команды}
Команды представляют из себя текстовые сообщения, в которых команда и ключ разделены пробелом.
\begin{center}\large\textbf{[CODE KEY]}\end{center}
Обмен сообщений происходит по следующему алгоритму:
\newline
Со стороны клиента:
\begin{enumerate}
\item Отправка длины сообщения
\item Отправка сообщения
\end{enumerate}
Со стороны серера:
\begin{enumerate}
\item Прием длины ожидаемого сообщения
\item Прием сообщения
\item Дешифрация сообщения
\item Выполнения команды
\end{enumerate}
\section{Прототипы функций команд:}
\begin{enumerate}
\item int HelpFunction(int command)
\item int CreateFunction(int command, char *argument)
\item int QuitFunction(int command)
\item int ForumsFunction(int command)
\item int OpenFunction(int command, char *argument)
\item int DisplayFunction(int command)
\item int OnlineFunction(int command)
\end{enumerate}

\section{Команды:}
\begin{enumerate}
\item help – получение полного списка команд с сервера
\item create [forum name] - создание нового форума с именем forum name
\item quit -  отключение от сервера
\item forums – получения списка доступных форумов
\item open [forum name] – открытие форума с именем forum name
\item --quit – выйти из форума
\item ie - запросить иерархию форума
\item online - запросить список пользователей онлайн
\end{enumerate}

\section{Описание}
Для запуска севера используется строка вида:
\begin{center}\large\textbf{[server.exe port]}\end{center}
где:
\newline
\textbf{server.exe} - скомпилированый исполняемый файл сервера
\newline
\textbf{port} - порт который будет прослушивать сервер
\newline
Для запуска клиента используется строка вида:
\begin{center}\large\textbf{[client.exe ipAdr port]}\end{center}
где:
\newline
\textbf{client.exe} - скомпилированый исполняемый файл клиента
\newline
\textbf{ipAdr} - адрес на котором запущен TCP-сервер
\newline
\textbf{port} - порт который будет прослушивать сервер
\newline
При первом запуске потребуется ввести имя пользователя, после прохождения проверки клиент будет подключен к серверу.
Максимальное количество поддерживаемых форумов:
\begin{verbatim}
\#define MAXFORUMS       1000
\end{verbatim}
Максимальное количество поддерживаемых клиентов:
\begin{verbatim}
\#define MAXCLIENTS      5
\end{verbatim}
Используемые библиотеки сервера:
\begin{verbatim}
\#include <winsock2.h>
\#include <windows.h>
\end{verbatim}
Тип сокета:
\begin{verbatim}
WSADATA wsa;
SOCKET serverSocket,accSocket;
\end{verbatim}

\end{document}